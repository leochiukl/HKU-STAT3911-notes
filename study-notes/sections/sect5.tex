\section{Risk-Neutral Pricing}
\label{sect:risk-neutral-pricing}
\begin{enumerate}
\item Equipped with the tools developed in \Cref{sect:stochastic-calculus}, we
are now able to study the mathematical details of an extremely powerful method
for pricing derivatives, known as \emph{risk-neutral pricing}, which lies in
the heart of modern financial economics. In STAT3905/STAT3910, we have
briefly investigated the risk-neutral pricing and carried out some numerical
computations about those risk-neutral pricing formulas; the mathematical
details about the risk-neutral pricing have not been fully discussed. In
\Cref{sect:risk-neutral-pricing}, we will delve into the mathematical details
and justify rigorously why the risk-neutral pricing works, using tools from
previous sections.
\item \textbf{Basic idea of risk-neutral pricing.} Mathematically, the method
of risk-neutral pricing is closely related to the notion of \emph{martingale}
(hence it is sometimes called \emph{martingale pricing}). The idea is to change
the probability measure (as suggested in \Cref{subsect:change-of-meas}) such
that a martingale is formed from the stock prices discounted by the risk-free
rate, \(e^{-rt}S_t\).  With \(\{e^{-rt}S_t\}\) being a
\(\{\mathcal{F}_t\}\)-martingale, we know that
\(\expv{e^{-rt}S_t}=\expv{\expv{e^{-rt}S_t|\mathcal{F}_0}}=\expv{e^{-0}S_0}=S_0\)
(assuming \(S_0\) is nonrandom), meaning that the future discounted stock price
\(e^{-rt}S_t\) can be ``passed'' to the present price \(S_0\) upon taking
expectation.

Rearranging this equation yields \(\expv{S_t}=S_0e^{rt}\), which tells us that
the mean return rate of the (risky) stock is still the risk-free rate. As such
mean return rate would arise under the \emph{risk-neutral} preference, we call
this pricing method as \underline{risk-neutral} pricing.

\begin{note}
In the real market (with actual probability measure), the mean return rate of
risky stock is usually higher than the risk-free rate (the difference is known
as the \emph{risk premium})), which corresponds to the case of
\emph{risk-averse} preference, under which extra return is demanded for
investing in the risky stock.
\end{note}
\end{enumerate}
\subsection{Risk-Neutral Measure}
\begin{enumerate}
\item\label{it:rn-meas-absorb-drift} \textbf{Absorbing drift terms through
changing to risk-neutral measure.} Intuitively, changing the probability
measure to a \emph{risk-neutral measure} can be interpreted as ``absorbing''
the drift term in the stock price process.

 To understand this better, consider a stock price process \(\{S_t\}\)
following a geometric Brownian motion: \(\odif{S_t}=\alpha S_t\odif{t} +\sigma
S_{t}\odif{W_t}\), where \(\{W_t\}\) is a Brownian motion, \(\alpha\in\R\), and
\(\sigma>0\). Let \(r\) be the risk-free rate and \(D_t=e^{-rt}\) be the
time-\(t\) discount factor for every \(t\). Then, by It\^o formula with
\(f(t,x)=e^{-rt}x=D_tx\), we have
\begin{align*}
\odif{(D_tS_t)}
&=f_{t}(t,S_t)\odif{t}+f_{x}(t,S_t)\odif{S_t}+0
=-rD_tS_t\odif{t}+D_t\odif{S_t} \\
&=-rD_tS_t\odif{t}+\alpha D_tS_t\odif{t} +\sigma D_t S_{t}\odif{W_t}
=\vc{D_tS_t(\alpha-r)}\odif{t}+\sigma D_tS_t\odif{W_t} \\
&=\sigma D_tS_t\left(\frac{\alpha-r}{\sigma}\odif{t}+\odif{W_t}\right)
\overset{(\widetilde{W}_t=\int_{0}^{t}\frac{\alpha-r}{\sigma}\odif{u}
+\int_{0}^{t}1\odif{W_u}=\frac{\alpha-r}{\sigma}t+W_t)}{=}
\sigma D_tS_t\odif{\widetilde{W}_t}.
\end{align*}
Under the original probability measure, \(\{\widetilde{W}_t\}\) would generally
\emph{not} be a Brownian motion, and so one cannot just apply
\labelcref{it:zero-drift-mart} to conclude that \(\{D_tS_t\}\) is martingale.
However, after changing the probability measure to a \emph{risk-neutral} one,
the process \(\{\widetilde{W}_t\}\) would indeed become a Brownian motion.
Therefore, in such case, changing the probability measure to the risk-neutral
measure does ``absorb'' the original drift term \(\vc{D_tS_t(\alpha-r)}\), without
influencing the volatility term (still \(\sigma D_t S_{t}\)). As the drift term
becomes zero, we can then apply \labelcref{it:zero-drift-mart} to conclude that
\(\{D_tS_t\}\) is a martingale \emph{under the risk-neutral measure}.

\item \textbf{Preliminaries.} The idea put forward in
\labelcref{it:rn-meas-absorb-drift} can be mathematically supported through the
\emph{Girsanov theorem}. Here we will lay some groundwork that will be helpful
in proving the Girsanov theorem. In the following, we are going to establish a
``conditional version'' of \Cref{prp:exp-change-of-meas}, which tells us the
effect on \emph{conditional} expectation after change of measure.

\begin{lemma}
\label{lma:cond-change-of-meas}
Let \(\{\mathcal{F}_t\}_{t\in [0,T]}\) be a filtration, \(Z\) be an almost
surely positive random variable satisfying \(\expv{Z}=1\). Let \(\tpr\) be
defined as in \Cref{eq:change-of-measure}, i.e., \(
\tprob{A}=\int_{A}^{}Z(\omega)\odif{\prob{\omega}}\) for every
\(A\in\mathcal{F}\). Define a process \(\{Z_t\}_{t\in [0,T]}\) by
\(Z_t=\expv{Z|\mathcal{F}_t}\) for every \(t\in [0,T]\) (which is a Doob
martingale).
\begin{enumerate}
\item For all \(t\in [0,T]\), \(\texpv{Y}=\expv{YZ_t}\), where \(Y\) is
\(\mathcal{F}_t\)-measurable and \(YZ_t\) is integrable.
\item For all \(0\le s\le t\le T\),
\(\texpv{Y|\mathcal{F}_s}=\expv{YZ_t|\mathcal{F}_s}/Z_s\), where \(Y\) is
\(\mathcal{F}_t\)-measurable and \(YZ_t\) is integrable.
\end{enumerate}
\end{lemma}
\begin{pf}
\begin{enumerate}
\item Note that
\[
\texpv{Y}\overset{\text{(\Cref{prp:exp-change-of-meas})}}{=}\expv{YZ}
=\expv{\expv{YZ|\mathcal{F}_t}}\overset{\text{(TOWIK)}}{=}\expv{Y\expv{Z|\mathcal{F}_t}}
=\expv{YZ_t}.
\]
\item With the assumption that \(Z>0\) almost surely, it can be shown that
\(Z_s=\expv{Z|\mathcal{F}_s}>0\) almost surely, so the (almost sure) equality is
well-defined. Next, we verify the two properties of conditional expectations:
\begin{enumerate}[label={(\arabic*)}]
\item Since \(\expv{YZ_t|\mathcal{F}_s}\) and \(Z_s\) are both
\(\mathcal{F}_s\)-measurable, \(\expv{YZ_t|\mathcal{F}_s}/Z_s\) is
\(\mathcal{F}_s\)-measurable.
\item Fix any \(A\in\mathcal{F}_s\). Then, 
\begin{align*}
\int_{A}^{}\frac{\expv{YZ_t|\mathcal{F}_s}}{Z_s}\odif{\tpr}
&=\texpv{\indic_{A}\frac{1}{Z_s}\expv{YZ_t|\mathcal{F}_s}}
\underset{(\times Z_s)}{\overset{\text{(a)}}{=}}
=\expv{\indic_{A}\expv{YZ_t|\mathcal{F}_s}}
\overset{\text{(TOWIK)}}{=}
\expv{\expv{\indic_{A}YZ_t|\mathcal{F}_s}} \\
\overset{\text{(iterated conditioning)}}&{=}\expv{\indic_{A}YZ_t}
\overset{\text{(a)}}{=}\texpv{\indic_{A}Y}
=\int_{A}^{}Y\odif{\tpr}.
\end{align*}
\end{enumerate}
\end{enumerate}
\end{pf}

\begin{note}
\Cref{lma:cond-change-of-meas} is typically applied in the following context.
Let \(\{Z_t\}\) be a \(\{\mathcal{F}_t\}\)-martingale, and \(Z=Z_T\). Assume
that \(Z=Z_T\) is an almost surely positive random variable satisfying
\(\expv{Z}=1\). Noting that
\(\expv{Z|\mathcal{F}_t}=\expv{Z_T|\mathcal{F}_t}=Z_t\) for all \(t\in [0,T]\)
in such case, we can then use \Cref{lma:cond-change-of-meas} to conclude that
\(\texpv{Y}=\expv{YZ_t}\) and
\(\texpv{Y|\mathcal{F}_s}=\expv{YZ_t|\mathcal{F}_s}/Z_s\), where \(Y\) is
\(\mathcal{F}_t\)-measurable, for all \(0\le s\le t\le T\).
\end{note}
\item \textbf{Girsanov theorem.} Now, we are ready to study the \emph{Girsanov
theorem}, which justifies the idea of ``absorbing drift term'' from
\labelcref{it:rn-meas-absorb-drift}.
\begin{theorem}[Girsanov]
\label{thm:girsanov}
Let \(\{W_t\}_{t\in[0,T]}\) be a Brownian motion on a probability space
\((\Omega,\mathcal{F},\pr)\), where \(T>0\) is a constant,
\(\{\mathcal{F}_t\}_{t\in [0,T]}\) be a filtration for the Brownian motion, and
\(\{\Theta_t\}_{t\in[0,T]}\) be a \(\{\mathcal{F}_t\}\)-adapted process. For
all \(t\in [0,T]\), define
\begin{align*}
Z_t&=\exp\left[-\int_{0}^{t}\Theta_u\odif{W_u}
-\frac{1}{2}\int_{0}^{t}\Theta_{u}^{2}\odif{u}\right],\\
\widetilde{W}_t&=W_t+\int_{0}^{t}\Theta_{u}\odif{u}.
\end{align*}
Suppose the square-integrability condition
\(\expv{\int_{0}^{T}\Theta_{u}^{2}Z_u^{2}\odif{u}}<\infty\) is satisfied. Let
\(Z=Z_T>0\) and define \(\tpr\) as in \Cref{eq:change-of-measure}, i.e., \(
\tprob{A}=\int_{A}^{}Z(\omega)\odif{\prob{\omega}}\) for every
\(A\in\mathcal{F}\). Then, \(\{\widetilde{W}_t\}_{t\in [0,T]}\) is a Brownian
motion on the probability space \((\Omega,\mathcal{F},\tpr)\).
\end{theorem}
\begin{pf}
\textbf{Showing that \(\{\widetilde{W}_t\}\) accumulates quadratic variations
at unit rate.} Note that we can write
\(\widetilde{W}_t=\int_{0}^{t}\vc{1}\odif{W_u}+\int_{0}^{t}\Theta_{u}\odif{u}\)
for all \(t\), and hence \(\{\widetilde{W}_t\}\) is an It\^o process (under
\(\pr\)). Thus, by \Cref{prp:ito-process-quad-var}, we know that
\(\odif{\widetilde{W}_t,\widetilde{W}_t}=\vc{1}^{2}\odif{t}=\odif{t}\).

\textbf{Showing that \(\{\widetilde{W}_tZ_t\}\) is a martingale under \(\pr\).}
Let \(X_t=-\int_{0}^{t}\Theta_u\odif{W_u}
-\frac{1}{2}\int_{0}^{t}\Theta_{u}^{2}\odif{u}\). Then, \(\{X_t\}\) is an It\^o
process and we have \(\odif{X_t}=-\frac{1}{2}\Theta_t^{2}\odif{t}-\Theta_t\odif{W_t}\).
Writing \(Z_t=e^{X_t}\), by the It\^o formula for It\^o process we have
\[
\odif{Z_t}=Z_t\odif{X_t}+\frac{1}{2}Z_t\odif{X_t}\odif{X_t}
=-\frac{1}{2}\Theta_t^{2}Z_t\odif{t}-\Theta_tZ_t\odif{W_t}+\frac{1}{2}Z_t\Theta_t^{2}\odif{t}
=-\Theta_tZ_t\odif{W_t}.
\]
Therefore, applying the It\^o product rule gives
\begin{align*}
\odif{\widetilde{W}_tZ_t}
&=\widetilde{W}_t\odif{Z_t}+Z_t\odif{\widetilde{W}_t}+\odif{\widetilde{W}_t,Z_t} \\
&=\widetilde{W_t}(-\Theta_tZ_t\odif{W_t})+Z_t(\odif{W_t}+\Theta_t\odif{t})
+(\odif{W_t}+\Theta_t\odif{t})(-\Theta_tZ_t\odif{W_t}) \\
&=(-\widetilde{W_t}\Theta_tZ_t+Z_t)\odif{W_t}.
\end{align*}
Since the drift term is zero, by \labelcref{it:zero-drift-mart} we conclude
that \(\{\widetilde{W}_tZ_t\}\) is a martingale under \(\pr\).

\textbf{Showing that \(\{\widetilde{W}_t\}\) is a martingale under \(\tpr\).}
It is straightforward to see that \(\widetilde{W}_t\) is both integrable and
\(\mathcal{F}_t\)-measurable for all \(t\in [0,T]\). Thus, it remains to show
that \(\texpv{\widetilde{W_t}|\mathcal{F}_s}=\widetilde{W_s}\) for all \(0\le
s\le t\le T\).  Fix any \(0\le s\le t\le T\). Since the drift term is zero in
the SDE \(\odif{Z_t}=-\Theta_tZ_t\odif{W_t}\), we know by
\labelcref{it:zero-drift-mart} that \(\{Z_t\}\) is a martingale under \(\pr\),
and so \(\expv{Z}=\expv{Z_T}=\expv{\expv{Z_T|\mathcal{F}_0}}=\expv{Z_0}=1\).
Hence, we can apply \Cref{lma:cond-change-of-meas} to get
\[
\texpv{\widetilde{W}_t|\mathcal{F}_s}
=\frac{1}{Z_s}\expv{\widetilde{W}_tZ_t|\mathcal{F}_s}
=\frac{1}{Z_s}\widetilde{W}_sZ_s
=\widetilde{W}_s.
\]
\textbf{Applying L\'evy's characterization theorem.} Observing that
\(\{\widetilde{W}_{t}\}\) starts at zero and has continuous paths, by
\Cref{thm:levy-char} we can then conclude that \(\{\widetilde{W}_{t}\}\) is a
Brownian motion on the probability space \((\Omega,\mathcal{F},\tpr)\).
\end{pf}

\item\label{it:multi-dim-rn-meas} \textbf{Risk-neutral measure.} The
probability measure \(\tpr\) from the Girsanov theorem is indeed a
\emph{risk-neutral measure}.  Like \labelcref{it:rn-meas-absorb-drift}, it can
be interpreted as being designed for absorbing the drift term. But here, rather
than working with only the geometric Brownian motion, we are considering a more
general setting with \emph{generalized} geometric Brownian motion.

Let \(\{W_t\}_{t\in [0,T]}\)
be a Brownian motion and \(\{\mathcal{F}_t\}_{t\in [0,T]}\) be a filtration for
the Brownian motion, where \(T>0\) is fixed. Suppose that the stock price
process \(\{S_t\}_{t\in [0,T]}\) follows a generalized geometric Brownian
motion:
\[
\odif{S_t}=\alpha_tS_t\odif{t}+\sigma_tS_t\odif{W_t}
\]
where \(\{\alpha_t\}\) and \(\{\sigma_{t}\}\) are \(\{\mathcal{F}_t\}\)-adapted
processes, with \(\sigma_{t}>0\) for all \(t\in [0,T]\). In integral form, we
can express it as
\[
S_t=S_0\exp\left[\int_{0}^{t}\sigma_u\odif{W_u}
+\int_{0}^{t}\left(\alpha_u-\frac{1}{2}\sigma_{u}^{2}\right)\odif{u}\right]\qquad
\text{for all \(t\in [0,T]\).}
\]
Furthermore, suppose that the risk-free rate process \(\{r_t\}\) is
\(\{\mathcal{F}_t\}\)-adapted.  For every \(t\in [0,T]\), let
\(D_t=e^{-\int_{0}^{t}r_u\odif{u}}\) be the time-\(t\) discount factor.
Then, we have
\[
D_tS_t=S_0\exp\left[\int_{0}^{t}\sigma_u\odif{W_u}
+\int_{0}^{t}\left(\alpha_u-r_u-\frac{1}{2}\sigma_{u}^{2}\right)\odif{u}\right].
\]
Now, applying It\^o formula with \(X_t
=\int_{0}^{t}\sigma_u\odif{W_u}
+\int_{0}^{t}\left(\alpha_u-r_u-\frac{1}{2}\sigma_{u}^{2}\right)\odif{u}\)
and \(f(t,x)=S_0e^{x}\) gives
\begin{align*}
\odif{(D_tS_t)}&=S_0e^{X_t}\odif{X_t}+\frac{1}{2}S_0e^{X_t}\sigma_t^{2}\odif{t}
=D_tS_t\left[(\alpha_t-r_t-\sigma_t^{2}/2)\odif{t}+\sigma_t\odif{W_t}\right]
+\frac{1}{2}D_tS_t\sigma_t^{2}\odif{t} \\
&=\sigma_tD_tS_t\left(\frac{\alpha_t-r_t}{\sigma_t}\odif{t}+\odif{W_t}\right)
=\sigma_tD_tS_t(\Theta_t\odif{t}+\odif{W_t})
\overset{(\widetilde{W}_t=W_t+\int_{0}^{t}\Theta_u\odif{u})}{=}
=\sigma_t D_tS_t\odif{\widetilde{W}_t},
\end{align*}
where \(\Theta_t=(\alpha_t-r_t)/\sigma_t\) is known as the \defn{market price of
risk} (which satisfies the square-integrability condition mentioned in the
Girsanov theorem).

Applying Girsanov theorem, we know that \(\{\widetilde{W}_t\}\) is a Brownian
motion under the probability measure \(\tpr\), and so the drift term of
\(\{D_tS_t\}\) is ``absorbed'' after changing the probability measure to \(\tpr\),
and \(\{D_tS_t\}\) is a martingale under the probability measure \(\tpr\).
In addition, by \Cref{prp:z-positive-equiv-prob}, we know that \(\tpr\) and
\(\pr\) are equivalent.

Mathematically, a probability measure \(\tpr\) is said to be a
\defn{risk-neutral measure} (or \defn{(equivalent) martingale measure}) if
\begin{enumerate}[label={(\arabic*)}]
\item \(\tpr\) and \(\pr\) are equivalent.
\item Under \(\tpr\), the discounted stock price process \(\{D_tS_t\}\) is a
\(\{\mathcal{F}_t\}\)-martingale.
\end{enumerate}
\begin{note}
In \labelcref{it:multi-dim-rn-meas}, we will revisit the concept of
risk-neutral measure in a more general setting.
\end{note}
\item \textbf{Dynamics under risk-neutral measure.} After studying the
theoretical foundation of risk-neutral measure, we are going to analyze the
behaviour of the prices of different assets under the risk-neutral measure, to
better understand the impacts on the market dynamics from changing the
probability measure to the risk-neural one.
\begin{enumerate}
\item \emph{(Stock)}
Combining the SDE \(\odif{S_t}=\alpha_tS_t\odif{t}+\sigma_tS_t\odif{W_t}\) for
the generalized geometric Brownian motion and the SDE
\(\odif{\widetilde{W}_t}=\Theta_t\odif{t}+\odif{W_t}\), we get
\[
\odif{S_t}=\alpha_tS_t\odif{t}+\sigma_tS_t(\odif{\widetilde{W}_t}-\Theta_t\odif{t})
=r_tS_t\odif{t}+\sigma_tS_t\odif{\widetilde{W}_t}.
\]
This suggests that under the risk-neutral measure \(\tpr\), the stock price
process \(\{S_t\}\) still follows a generalized geometric Brownian motion with
the drift term changed to \(r_tS_t\) (mean return being risk-free rate)
and the volatility term unchanged.
\item\label{it:self-fin-port-rn} \emph{(Self-financing portfolio)}
Consider a self-financing portfolio analogous to the one in
\labelcref{it:construct-self-fin-port}, but for the case of generalized
geometric Brownian motion here.

Like \labelcref{it:construct-self-fin-port}, at each time \(t\), let \(X_t\)
denote the investor's portfolio value and \(\Delta_t\) denote the number of
shares of stock held by the investor. Suppose that \(\{\Delta_t\}\) is
adapted to a filtration \(\{\mathcal{F}_t\}\) for the Brownian motion
\(\{W_t\}\), and at each time \(t\), the non-stock component of the portfolio,
with value \(X_t-\Delta_t S_t\), is all invested in the bond.

Then, the ``change'' \(\odif{X}_t\) for investor's portfolio at time \(t\)
sources from (i) the gain \(\Delta_t\odif{S_t}\) on the stock position and (ii) the
interest earning \(r_{\vc{t}}(X_t-\Delta_tS_t)\odif{t}\) on the bond position.
Hence, we can write
\begin{align*}
\odif{X_t}&=\Delta_t\underbrace{\odif{S_t}}_{\mathclap{\alpha_tS_t\odif{t}+\sigma_tS_t\odif{W_t}}}
+r_t(X_t-\Delta_tS_t)\odif{t}
=r_tX_t\odif{t}+\Delta_t(\alpha_t-r_t)S_t\odif{t}+\Delta_t\sigma_tS_t\odif{W_t} \\
&=r_tX_t\odif{t}+\Delta_t\sigma_tS_t(\Theta_t\odif{t}+\odif{W_t})
\overset{(\widetilde{W}_t=W_t+\int_{0}^{t}\Theta_u\odif{u})}{=}
r_tX_t\odif{t}+\Delta_t\sigma_t S_t\odif{\widetilde{W}_t}
\end{align*}
where \(\Theta_t=(\alpha_t-r_t)/\sigma_t\). This suggests that under the
risk-neutral measure, the (undiscounted) portfolio values also follow a
generalized geometric Brownian motion, where the mean return is still risk-free
rate like the stock, but the volatility term differs and depends on the stock
position \(\Delta_t\).

Further insights can be gained by studying the behaviour of \emph{discounted}
portfolio values. Again, let \(D_t=e^{-\int_{0}^{t}r_u\odif{u}}\) be the
time-\(t\) discount factor for every \(t\). Applying It\^o formula with
\(X_t=\int_{0}^{t}r_u\odif{u}\) and \(f(t,x)=e^{-x}\) gives
\[
\odif{D_t}=-\underbrace{e^{-X_t}}_{D_t}\underbrace{\odif{X_t}}_{r_t\odif{t}}+0
=-r_tD_t\odif{t}.
\]
Hence, by It\^o product rule, we have
\begin{align*}
\odif{(D_tX_t)}&=D_t\odif{X_t}+X_t\odif{D_t}+\odif{D_t}\odif{X_t}
=D_t(r_tX_t\odif{t}+\Delta_t\sigma_t S_t\odif{\widetilde{W}_t})
+X_t(-r_tD_t\odif{t})+0 \\
&=\Delta_t\underbrace{\sigma_t D_tS_t\odif{\widetilde{W}_t}}_{\odif{D_tS_t}}
=\Delta_t\odif{(D_tS_t)}.
\end{align*}
This shows that \(\{D_tX_t\}\) is a martingale under the risk-neutral measure,
like the stock prices. Furthermore, the SDE above indicates that the ``change''
in discounted portfolio value \(\odif{(D_tX_t)}\) indeed only sources from the
``discounted gain'' \(\Delta_t\odif{(D_tS_t)}\) on the stock position.
\item\label{it:rn-pricing} \emph{(Derivative)} In the proof of
\Cref{thm:bs-eqn}, we have utilized a replication argument for an European call
option. Here, we extend this idea to \emph{any} derivative that can be
replicated; we shall still impose the usual assumption that the market is
arbitrage-free, with a stock and a (risk-free) bond which can be freely bought
or (short) sold in any amount without transaction cost.

\begin{note}
We will have more discussions on the existence of replicating portfolio in
\labelcref{it:one-stock-rep}.
\end{note}

Consider a \(T\)-year derivative with time-\(T\) payoff \(V_T\). Under the
assumption that it can be replicated, there is a self-financing portfolio with
price process \(\{X_t\}\) such that \(X_T=V_T\). Under the no-arbitrage
condition, this implies that \(X_t=V_t\) for all \(0\le t<T\) also, where
\(V_t\) denotes the time-\(t\) price of the derivative.

After having such self-financing portfolio constructed, by
\labelcref{it:self-fin-port-rn} we know that \(\{D_tX_t\}\) is a martingale
under the risk-neutral measure \(\tpr\). Thus, for every \(t\in [0,T]\), we
have
\[
D_tV_t\overset{(X_t=V_t)}{=}D_tX_t=\texpv{D_TX_T|\mathcal{F}_t}
\overset{(X_T=V_T)}{=}\texpv{D_TV_T|\mathcal{F}_t}.
\]
Since \(D_T/D_t=e^{-\int_{0}^{T}r_u\odif{u}}/e^{-\int_{0}^{t}r_u\odif{u}}
=e^{-\int_{t}^{T}r_u\odif{u}}\), rearranging the equation gives
\[
V_t=\texpv{\left. e^{-\int_{t}^{T}r_u\odif{u}}V_T\right|\mathcal{F}_t}
\quad\text{for all \(t\in [0,T]\),}
\]
which is the \emph{risk-neutral pricing formula}.
\end{enumerate}
\item \textbf{Deriving Black-Scholes equation and Black-Scholes formula through
risk-neutral measure.}
In \Cref{subsect:bs-eqn}, we have derived the Black-Scholes equation and
formula purely by stochastic calculus. However, this is not the only way to do
so, and the notion of \emph{risk-neutral measure} gives us an alternative route
for deriving them. Let \(V(t,S_t)\) denote the time-\(t\) value of a \(T\)-year
derivative satisfying the assumptions specified in \Cref{thm:bs-eqn}.
Assume that \(r_t=r\) and \(\sigma_t=\sigma\) for all \(t\in [0,T]\), where
\(r\) and \(\sigma>0\) are constants.
\begin{enumerate}
\item \emph{(Deriving Black-Scholes formula)} Previously the Black-Scholes
formula is derived as the solution to the Black-Scholes equation, for European
call. A more direct way of deriving the formula is to utilize the
\emph{risk-neutral pricing formula} from \labelcref{it:rn-pricing}; such
approach is indeed used in STAT3905/STAT3910, where the Black-Scholes formula
is derived through suitable algebraic manipulations. The key idea is as
follows. Suppose the derivative in consideration is a \(T\)-year \(K\)-strike
European call.
\[
V(t,S_t)\overset{\text{\labelcref{it:rn-pricing}}}{=}\expv{e^{-r(T-t)}(S_T-K)_{+}|\mathcal{F}_t}
\overset{\text{(algebra)}}{=}
S_t\Phi(d_{+}(T-t,S_t))-Ke^{-r(T-t)}\Phi(d_{-}(T-t,S_t)).
\]
\item\label{it:deriv-bs-eq-rn} \emph{(Deriving Black-Scholes equation)} The
idea of risk-neutral measure also leads to an alternative method for deriving
the Black-Scholes equation, by working with the differentials directly and
applying the converse of \labelcref{it:zero-drift-mart}.

Consider the process \(\{V(t,S_t)\}\) of derivative prices, and its corresponding
replicating portfolio \(\{X_t\}\), satisfying that \(X_t=V(t,S_t)\) for all
\(t\in[0,T]\). From the proof of \Cref{thm:bs-eqn}, we know
\begin{align*}
\odif{{V(t,S_t)}}
&=\left[V_{t}(t,S_t)+\alpha S_tV_{x}(t,S_t)
+\frac{1}{2}\sigma^{2}S_{t}^{2}V_{xx}(t,S_t)\right]\odif{t}
+\sigma S_{t}V_{x}(t,S_t)\odif{W_t} \\
\overset{(\widetilde{W}_t=\frac{\alpha-r}{\sigma}t+W_t)}&{=}
\left[V_{t}(t,S_t)+\alpha S_tV_{x}(t,S_t)
+\frac{1}{2}\sigma^{2}S_{t}^{2}V_{xx}(t,S_t)\right]\odif{t}
+\sigma S_{t}V_{x}(t,S_t)\left(\odif{\widetilde{W}_t}-\frac{\alpha-r}{\sigma}\odif{t}\right) \\
&=\left[V_{t}(t,S_t)+\vc{r}S_tV_{x}(t,S_t)
+\frac{1}{2}\sigma^{2}S_{t}^{2}V_{xx}(t,S_t)\right]\odif{t}
+\sigma S_{t}V_{x}(t,S_t)\odif{\widetilde{W}_t}.
\end{align*}
Hence, letting \(D_t=e^{-rt}\) denote the time-\(t\) discount factor and
applying It\^o product rule, we have
\begin{align*}
\odif{{(D_tV(t,S_t))}}
&=V(t,S_t)\underbrace{\odif{D_t}}_{\mathclap{-rD_t\odif{t}}}
+D_t\odif{{V(t,S_t)}}+\underbrace{\odif{D_t}\odif{{V(t,S_t)}}}_{0} \\
&=-rD_tV(t,S_t)\odif{t}+D_t\left\{
\left[V_{t}(t,S_t)+rS_tV_{x}(t,S_t)
+\frac{1}{2}\sigma^{2}S_{t}^{2}V_{xx}(t,S_t)\right]\odif{t}
+\sigma S_{t}V_{x}(t,S_t)\odif{\widetilde{W}_t}\right\} \\
&=D_t\left[V_{t}(t,S_t)+rS_tV_{x}(t,S_t)
+\frac{1}{2}\sigma^{2}S_t^{2}V_{xx}(t,S_t)-rV(t,S_t)\right]\odif{t}
+\sigma D_tS_{t}V_{x}(t,S_t)\odif{\widetilde{W}_t}.
\end{align*}
By \labelcref{it:self-fin-port-rn}, we know that \(\{D_tX_t\}=\{D_tV(t,S_t)\}\)
is a martingale. Therefore, applying the converse of
\labelcref{it:zero-drift-mart} gives
\[
V_{t}(t,S_t)+rS_tV_{x}(t,S_t)
+\frac{1}{2}\sigma^{2}S_t^{2}V_{xx}(t,S_t)-rV(t,S_t)=0.
\]
Rearranging this yields the Black-Scholes equation.

\end{enumerate}
\end{enumerate}
\subsection{Martingale Representation Theorem}
\begin{enumerate}
\item \textbf{Martingale representation theorem.} Recall from
\labelcref{it:ito-int-gen-prop} that It\^o integral forms a martingale. It
turns out that we can also work backwards, and establish that a
\underline{martingale} can be \underline{represented} in terms of an It\^o
integral, as suggested by the \emph{martingale representation theorem}.
\begin{theorem}[Martingale representation]
\label{thm:mart-rep}
Let \(\{W_t\}\) be a Brownian motion on a probability space
\((\Omega,\mathcal{F},\pr)\), and let \(\{\mathcal{F}_t\}\) be the filtration
generated by \(\{W_t\}\)\footnote{This means that for all \(t\),
\(\mathcal{F}_t\) is the smallest \(\sigma\)-algebra for which \(W_s\) is
\(\mathcal{F}_t\)-measurable for all \(0\le s\le t\).  Intuitively, this means
\(\mathcal{F}_t\) contains only the information arising from the Brownian
motion up to time \(t\). Particularly, \(\{\mathcal{F}_t\}\) is a special kind
of filtration for Brownian motion. For a more formal definition, see
STAT7610.}. If \(\{M_t\}\) is a
\(\{\mathcal{F}_t\}\)-martingale under \(\pr\), then there is a
\(\{\mathcal{F}_t\}\)-adapted process \(\{\Gamma_u\}\) such that \(
M_t=M_0+\int_{0}^{t}\Gamma_u\odif{W_u}\) for all \(t\).
\end{theorem}

\begin{pf}
Omitted.
\end{pf}

There is also a version of martingale representation theorem for the
risk-neutral measure as follows.
\begin{theorem}[Martingale representation (risk-neutral)]
\label{thm:mart-rep-rn}
Let \(\{W_t\}\) be a Brownian motion on a probability space
\((\Omega,\mathcal{F},\pr)\), and let \(\{\mathcal{F}_t\}\) be the filtration
generated by \(\{W_t\}\). Consider the risk-neutral probability measure
\(\tpr\) and the corresponding Brownian motion \(\{\widetilde{W}_t\}\) as
specified in the Girsanov theorem (assuming the conditions there hold). If
\(\{\widetilde{M}_t\}\) is a \(\{\mathcal{F}_t\}\)-martingale under \(\tpr\),
then there is a \(\{\mathcal{F}_t\}\)-adapted process
\(\{\widetilde{\Gamma}_u\}\) such that
\(\widetilde{M}_t=\widetilde{M}_0+\int_{0}^{t}\widetilde{\Gamma}_u\odif{\widetilde{W}_u}\)
for all \(t\).
\end{theorem}

\begin{note}
This does not immediately follow from \Cref{thm:mart-rep} since the
filtration \(\{\mathcal{F}_t\}\) is generated by the Brownian motion
\(\{W_t\}\) on \((\Omega,\mathcal{F},\pr)\), rather than the Brownian motion
\(\{\widetilde{W}_t\}\) on \((\Omega,\mathcal{F},\tpr)\).
\end{note}

\begin{pf}
Omitted.
\end{pf}
\item\label{it:one-stock-rep} \textbf{Replication with one stock.} We are now
equipped with enough tools to show that the existence of replication portfolio
as specified in \labelcref{it:rn-pricing}. It turns out that a requirement for
the existence is that the underlying filtration \(\{\mathcal{F}_t\}\) needs to
be the one generated by the Brownian motion \(\{W_t\}\), i.e., there
should not be ``extra'' information available other than the ones coming from
Brownian motion.

\begin{proposition}
\label{prp:rep-port-exist}
Let \(\{W_t\}\) be a Brownian motion on a probability space
\((\Omega,\mathcal{F},\pr)\), and let \(\{\mathcal{F}_t\}\) be the filtration
generated by \(\{W_t\}\). Let \(T>0\) and consider any derivative with
time-\(T\) payoff \(V_T\), where \(V_T\) is integrable (with respect to the
risk-neutral measure \(\tpr\)) and \(\mathcal{F}_T\)-measurable\footnote{This
condition just means that the payoff can be determined based on the information
at time \(T\), which is satisfied for derivatives with \(T\)-year term.}. Then,
the derivative can be replicated, by a self-financing portfolio with
\(X_0=\texpv{D_TV_T|\mathcal{F}_0}\) and
\(\Delta_t=\widetilde{\Gamma}_t/(\sigma_tD_tS_t)\) for all \(t\in [0,T]\),
where \(\widetilde{\Gamma}_t\) comes from \Cref{thm:mart-rep-rn}, and the other
notations carry their usual meanings.
\end{proposition}
\begin{pf}
Let \(\widetilde{M}_t=\texpv{D_TV_T|\mathcal{F}_t}\) for all \(t\in[0,T]\).
Note that \(\{\widetilde{M}_t\}\) is a Doob martingale under \(\tpr\). Thus, by
\Cref{thm:mart-rep-rn} we have
\[
\widetilde{M}_T=\widetilde{M}_0+\int_{0}^{T}\widetilde{\Gamma}_t\odif{\widetilde{W}_t}
\]
where \(\{\widetilde{\Gamma}_t\}\) is a \(\{\mathcal{F}_t\}\)-adapted process.

Next, consider any self-financing portfolio with time-\(t\) value \(X_t\) and
time-\(t\) stock position \(\Delta_t\) (with \(\{\Delta_t\}\) being
\(\{\mathcal{F}_t\}\)-adapted and the non-stock component being invested in the
risk-free bond). By \labelcref{it:self-fin-port-rn}, we know that
\(\{D_tX_t\}\) follows the SDE
\(\odif{D_tX_t}=\Delta_t\sigma_tD_tS_t\odif{\widetilde{W}_t}\). This implies
that
\[
D_TX_T=\underbrace{D_0}_{1}X_0+\int_{0}^{T}\Delta_t\sigma_tD_tS_t\odif{\widetilde{W}_t}
=X_0+\int_{0}^{T}\Delta_t\sigma_tD_tS_t\odif{\widetilde{W}_t}.
\]
Hence, by setting \(X_0=\widetilde{M}_0=\expv{D_TV_T|\mathcal{F}_0}\) and
\(\Delta_t=\widetilde{\Gamma}_t/(\sigma_tD_tS_t)\) for all \(t\in [0,T]\), we
can ensure that
\(D_TX_T=\widetilde{M}_{T}=\texpv{D_TV_T|\mathcal{F}_T}\overset{\text{(TOWIK)}}{=}D_TV_T\).
This implies that \(X_T=V_T\), and so a replicating portfolio is constructed.
\end{pf}

\begin{note}
While \Cref{prp:rep-port-exist} assures the \emph{existence} of replicating
portfolio under such conditions, the explicit construction of such portfolio
remains unclear since \Cref{thm:mart-rep-rn} does not give us a method for
finding \(\widetilde{\Gamma}_t\).
\end{note}
\end{enumerate}
\subsection{Market With Multiple Stocks}
\label{subsect:mult-stocks}
\begin{enumerate}
\item So far, we have exclusively worked in an arbitrage-free market consisting
of a \emph{single} stock and a (risk-free) bond which can be freely bought or
(short) sold in any amount without transaction cost, and studied the
replication of a \emph{single} derivative under such market. In
\Cref{subsect:mult-stocks}, we will generalize the idea and allow the market
to have more than one stock (which can all be transacted freely without cost
still), through which \emph{multiple} derivatives (whose payoffs may depend on
more than one stock) can be replicated.
\item \textbf{Multidimensional Girsanov theorem and martingale representation theorem.}
Like before, the \emph{Girsanov theorem} and the \emph{martingale
representation theorem} serve as two important tools for justifying the
replication argument here. However, since we are working in a multidimensional
market (having multiple stocks), we need the \emph{multidimensional version} of
these results. They will be stated without proof below.

\begin{theorem}[Girsanov (multidimensional)]
\label{thm:multi-dim-girsanov}
Fix a constant \(T>0\).
Let \(\{\vect{W}_{t}\}_{t\in [0,T]}\) be a \(d\)-dimensional Brownian motion on
a probability space \((\Omega,\mathcal{F},\pr)\), with
\(\vect{W}_t=(W_{t}^{(1)},\dotsc,W_{t}^{(d)})\) for all \(t\),
\(\{\mathcal{F}_t\}_{t\in [0,T]}\) be a filtration for the \(d\)-dimensional
Brownian motion, and \(\{\vect{\Theta}_t\}_{t\in[0,T]}\) be a \(d\)-dimensional
\(\{\mathcal{F}_t\}\)-adapted process, with
\(\vect{\Theta}_t=(\Theta_{t}^{(1)},\dotsc,\Theta_{t}^{(d)})\) for all \(t\).
For all \(t\in [0,T]\), define
\begin{align*}
Z_t&=\exp\left[-\int_{0}^{t}\vect{\Theta}_u\cdot \odif{\vect{W}_u}
-\frac{1}{2}\int_{0}^{t}\|\vect{\Theta}_u\|^{2}\odif{u}\right],\\
\widetilde{\vect{W}}_t&=\vect{W}_t+\int_{0}^{t}\vect{\Theta}_{u}\odif{u},
\end{align*}
where \(\int_{0}^{t}\vect{\Theta}_u\cdot \odif{\vect{W}_u}
:=\sum_{j=1}^{d}\int_{0}^{t}\Theta_u^{(j)}\odif{W_u^{(j)}}
\) \emph{(``dot product'' notation)} and \(\int_{0}^{t}\vect{\Theta}_u\odif{u}
:=(\int_{0}^{t}\Theta_u^{(1)}\odif{u},\dotsc,\int_{0}^{t}\Theta_u^{(d)}\odif{u})\)
\emph{(``vectorized'' notation)}.

Suppose the square-integrability condition
\(\expv{\int_{0}^{T}\|\Theta_{u}\|^{2}Z_u^{2}\odif{u}}<\infty\) is satisfied. Let
\(Z=Z_T>0\) and define \(\tpr\) as in \Cref{eq:change-of-measure}, i.e., \(
\tprob{A}=\int_{A}^{}Z(\omega)\odif{\prob{\omega}}\) for every
\(A\in\mathcal{F}\). Then, \(\{\widetilde{\vect{W}}_t\}_{t\in [0,T]}\) is a
\(d\)-dimensional Brownian motion on the probability space
\((\Omega,\mathcal{F},\tpr)\).
\end{theorem}
\begin{pf}
Omitted.
\end{pf}
\begin{theorem}[Martingale representation (multidimensional)]
\label{thm:multi-dim-mart-rep}
Let \(\{\vect{W}_t\}\) be a \(d\)-dimensional Brownian motion on a probability
space \((\Omega,\mathcal{F},\pr)\), and let \(\{\mathcal{F}_t\}\) be the
filtration generated by \(\{\vect{W}_t\}\). If \(\{M_t\}\) is a
\(\{\mathcal{F}_t\}\)-martingale under \(\pr\), then there is a
\(\{\mathcal{F}_t\}\)-adapted and \(d\)-dimensional process
\(\{\vect{\Gamma}_u\}\), with \(\vect{\Gamma}_u=(\Gamma_{u}^{(1)},\dotsc,\Gamma_{u}^{(d)})\)
for all \(u\), such that \(M_t=M_0+\int_{0}^{t}\vect{\Gamma}_u\cdot
\odif{\vect{W}_u}\) for all \(t\).

Consider the probability measure \(\tpr\) and the
corresponding Brownian motion \(\{\widetilde{\vect{W}}_t\}\) as specified in
the multidimensional Girsanov theorem (\Cref{thm:multi-dim-girsanov}) (assuming
the conditions there hold).  If \(\{\widetilde{M}_t\}\) is a
\(\{\mathcal{F}_t\}\)-martingale under \(\tpr\), then there is a
\(\{\mathcal{F}_t\}\)-adapted and \(d\)-dimensional process
\(\{\widetilde{\vect{\Gamma}}_u\}\), with
\(\widetilde{\vect{\Gamma}}_u=(\widetilde{\Gamma}_{u}^{(1)},\dotsc,
\widetilde{\Gamma}_{u}^{(d)})\) for all \(u\), such that
\(\widetilde{M}_t=\widetilde{M}_0+\int_{0}^{t}\widetilde{\vect{\Gamma}}_u\cdot
\odif{\widetilde{\vect{W}}_u}\) for all \(t\).

\end{theorem}
\item\label{it:market-dynamics} \textbf{Assumptions on the market dynamics.} In
addition to the usual no-arbitrage and free-transaction assumptions on the
multidimensional market, we will impose the following assumptions on the market
dynamics (i.e., the behaviours of the price processes).

Let \(\{\vect{W}_t\}_{t\in [0,T]}\) be a \(d\)-dimensional Brownian motion on a
probability space \((\Omega,\mathcal{F},\pr)\), associated with a filtration
\(\{\mathcal{F}_t\}_{t\in [0,T]}\), where \(T>0\) is a constant. Suppose that
\(\mathcal{F}=\mathcal{F}_T\), and there are \(m\) stocks, whose time-\(t\)
prices are \(S_t^{(1)},\dotsc,S_t^{(m)}\) respectively, driven by the following
SDEs:
\begin{equation}
\label{eq:mult-stock-sde}
\odif{S_{t}^{(i)}}
=\alpha_t^{(i)}S_t^{(i)}\odif{t}+S_{t}^{(i)}\sum_{j=1}^{d}\sigma_{t}^{(ij)}\odif{W_{t}^{(j)}}
,\quad i=1,\dotsc,m,
\end{equation}
where \(\{(\alpha_t^{(1)},\dotsc,\alpha_t^{(m)})\}_{t}\) and \(\{
[\sigma_{t}^{(ij)}]_{i,j=1}^{m,d}\}_{t}\) are
\(\{\mathcal{F}_t\}\)-adapted\footnote{Intuitively, this means that at each
time \(t\) these vectors and matrices can be determined based on the
information from \(\{\mathcal{F}_t\}\). We will not give a formal definition here;
see STAT7610 if interested.}, with \(\sigma_{t}^{(ij)}\ge 0\) and
\(\sum_{j=1}^{d}\sigma_{t}^{(ij)}>0\)
for all \(t\). Furthermore, we suppose the risk-free rate process \(\{r_t\}\) is
\(\{\mathcal{F}_t\}\)-adapted.

\begin{note}
The parameters can be interpreted as follows. The value \(\alpha_t^{(i)}\)
refers to the mean return rate of stock \(i\) at time \(t\), and the value
\(\sigma_t^{(ij)}\) refers to the volatility contribution from the \(j\)th
component of Brownian motion to the stock \(i\).
\end{note}

\item \textbf{Analysis of the stock price behaviours.}
\begin{enumerate}
\item \emph{(Generalized geometric Brownian motion for each stock)} For every
\(i=1,\dotsc,m\), define a process \(\{B_t^{(i)}\}\) by \[
B_t^{(i)}=\sum_{j=1}^{d}\int_{0}^{t}\frac{\sigma_{u}^{(ij)}}{\sigma_{u}^{(i)}}\odif{W_{u}^{(j)}}
\]
where \(\sigma_u^{(i)}:=\sqrt{\sum_{j=1}^{d}(\sigma_{u}^{(ij)})^{2}}>0\). In
differential form, we can write \(\odif{B_t^{(i)}}=\sum_{j=1}^{d}
(\sigma_{t}^{(ij)})/(\sigma_{t}^{(i)})\odif{W_{t}^{(j)}}\). As a sum of It\^o
integrals, by \labelcref{it:ito-int-gen-prop} and
\Cref{prp:lin-comb-mart-is-mart} we know that each \(\{B_t^{(i)}\}\) has
continuous paths and is a \(\{\mathcal{F}_t\}\)-martingale. Moreover, we have
\(B_0^{(i)}=0\) and
\[
\odif{B_{t}^{(i)},B_{t}^{(i)}}=\sum_{j=1}^{d}
\frac{(\sigma_{t}^{(ij)})^{2}}{(\sigma_{t}^{(i)})^{2}}\odif{t}
=\odif{t}
\]
for each \(i=1,\dotsc,m\). Therefore, by L\'evy's characterization theorem
(\Cref{thm:levy-char}), we conclude that each \(\{B_t^{(i)}\}\) is a Brownian motion.
Using \(B_t^{(i)}\), we can rewrite the SDEs in \labelcref{eq:mult-stock-sde} as
\[
\odif{S_{t}^{(i)}}
=\alpha_t^{(i)}S_t^{(i)}\odif{t}
+\sigma_t^{(i)}S_{t}^{(i)}\odif{B_{t}^{(i)}}
,\quad i=1,\dotsc,m,
\]
indicating that each stock indeed follows a generalized geometric Brownian motion.
\item \emph{(Relationship between the sources of randomness)} This representation of
SDEs also allows us to better understand the relationships between the sources
of randomness for different stocks as follows. Fix any \(i\ne k\). Then we have
\[
\odif{B_{t}^{(i)},B_{t}^{(k)}}
=\sum_{j=1}^{d}\frac{\sigma_{t}^{(ij)}\sigma_{t}^{(kj)}}
{\sigma_{t}^{(i)}\sigma_{t}^{(k)}}\odif{t}
=\rho_{t}^{(ik)}\odif{t}
\]
where \(\rho_{t}^{(ik)}:=\frac{1}{\sigma_{t}^{(i)}\sigma_{t}^{(k)}}
\sum_{j=1}^{d}\sigma_{t}^{(ij)}\sigma_{t}^{(kj)}\). By It\^o product rule, we have
\[
\odif{B_{t}^{(i)}B_{t}^{(k)}}
=B_{t}^{(i)}\odif{B_{t}^{(k)}}
+B_{t}^{(k)}\odif{B_{t}^{(i)}}+\odif{B_{t}^{(i)}B_{t}^{(k)}},
=B_{t}^{(i)}\odif{B_{t}^{(k)}}
+B_{t}^{(k)}\odif{B_{t}^{(i)}}
+\rho_{t}^{(ik)}\odif{t}
\]
which can be expressed in integral form as
\[
B_{t}^{(i)}B_{t}^{(k)}=\int_{0}^{t}B_{u}^{(i)}\odif{B_{u}^{(k)}}
+\int_{0}^{t}B_{u}^{(k)}\odif{B_{u}^{(i)}}
+\int_{0}^{t}\rho_{u}^{(ik)}\odif{u}.
\]
Upon taking expectations, we get the following covariance formula
\[
\cov{B_{t}^{(i)},B_{t}^{(k)}}=\expv{\int_{0}^{t}\rho_{u}^{(ik)}\odif{u}},
\]
since the expectation of It\^o integral is zero.
\item\label{it:multi-dim-disc-stock} \emph{(Dynamics of discounted stock
prices)} Like before, in the replication argument we often need to deal with
\emph{discounted} prices. So, here we will briefly investigate the behaviour of
the discounted stock prices.  Let \(D_t=e^{-\int_{0}^{t}r_u\odif{u}}\) be the
time-\(t\) discount factor, which satisfies that
\(\odif{D_t}=-r_t\odif{D_t}\odif{t}\).  Fix any \(i=1,\dotsc,m\), and consider
the discounted stock price process \(\{D_tS_t^{(i)}\}\). By It\^o product rule,
we have
\begin{align*}
\odif{(D_tS_{t}^{(i)})}&=D_t\odif{S_t^{(i)}}+S_{t}^{(i)}\odif{D_{t}}+
\underbrace{\odif{D_t,S_{t}^{(i)}}}_{0} \\
&=D_t\left(\alpha_t^{(i)}S_t^{(i)}\odif{t}+S_{t}^{(i)}\sum_{j=1}^{d}\sigma_{t}^{(ij)}\odif{W_{t}^{(j)}}\right)
+S_t^{(i)}(-r_tD_t\odif{t}) \\
&=D_tS_{t}^{(i)}\left[(\alpha_{t}^{(i)}-r_{t})\odif{t}
+\sum_{j=1}^{d}\sigma_{t}^{(ij)}\odif{W_{t}^{(j)}}\right].
\end{align*}
\end{enumerate}
\item\label{it:multi-dim-rn-meas} \textbf{Risk-neutral measure.}
In the multidimensional market here, we have a more general definition of
risk-neutral measure. A probability measure \(\tpr\) is said to be a
\defn{risk-neutral measure} (or \defn{(equivalent) martingale measure}) if
\begin{enumerate}[label={(\arabic*)}]
\item \(\tpr\) and \(\pr\) are equivalent.
\item Under \(\tpr\), the discounted stock price process \(\{D_tS_t^{(i)}\}\)
is a \(\{\mathcal{F}_t\}\)-martingale, for every \(i=1,\dotsc,m\).
\end{enumerate}
To investigate the second condition that \(\{D_tS_t^{(i)}\}\) is a
\(\{\mathcal{F}_t\}\)-martingale for every \(i=1,\dotsc,m\), it is helpful to
write the SDE from \labelcref{it:multi-dim-disc-stock} as follows:
\[
\odif{(D_tS_t^{(i)})}
=D_tS_{t}^{(i)}
\sum_{j=1}^{d}\sigma_{t}^{(ij)}(\Theta_{t}^{(j)}\odif{t}+\odif{W_{t}^{(j)}})
\overset{(\widetilde{W}_{t}^{(j)}=W_{t}^{(j)}+\int_{0}^{t}\Theta_{u}^{(j)}\odif{u})}{=}
D_tS_{t}^{(i)}
\sum_{j=1}^{d}\sigma_{t}^{(ij)}\odif{\widetilde{W}_{t}^{(j)}}
\]
where the processes \(\{\Theta_{t}^{(j)}\}\)'s satisfy \(\alpha_{t}^{(i)}-r_t
=\sum_{j=1}^{d}\sigma_{t}^{(ij)}\Theta_{t}^{(j)}\) for all \(i=1,\dotsc,m\),
known as the \defn{market price of risk equations}.

It can be shown that those processes \(\{\Theta_{t}^{(j)}\}\)'s exist (i.e., a
solution to the market price of risk equations exists) iff there is no
arbitrage in the market.

\begin{note}
The ``\(\Rightarrow\)'' direction can be established by the first fundamental
theorem of asset pricing (\Cref{thm:first-ftap}).
\end{note}

Hence, under the no-arbitrage assumption, we can then apply the
multidimensional Girsanov theorem to obtain a probability measure \(\tpr\) that
is equivalent to \(\pr\), under which \(\{\widetilde{\vect{W}}_t\}\) is a
\(d\)-dimensional Brownian motion, hence \(\{D_tS_{t}^{(i)}\}\) is a martingale
under \(\tpr\) for all \(i=1,\dotsc,m\). Therefore, such probability measure
\(\tpr\) is a risk-neutral measure. This suggests a method for constructing a
risk-neutral measure in our market.
\item \textbf{Dynamics of self-financing portfolio under risk-neutral measure.}
It turns out that, like \labelcref{it:self-fin-port-rn}, the discounted prices
of a self-financing portfolio always form a martingale under \(\tpr\). This
result is proved below and serves as a lemma for an important result in
financial economics, known as the \emph{first fundamental theorem of asset
pricing}.

\begin{lemma}
\label{lma:disc-port-price-mart-rn}
Let \(\tpr\) be a risk-neutral measure and \(X_t\) be the time-\(t\) value of a
self-financing portfolio, with time-\(t\) position on stock \(i\) being
\(\Delta_{t}^{(i)}\), where \(\{\Delta_{t}^{(i)}\}\) is
\(\{\mathcal{F}_t\}\)-adapted for each \(i=1,\dotsc,m\), and the non-stock
component being invested in the risk-free bond. Then, the discounted portfolio
value process \(\{D_tX_t\}\) is a \(\{\mathcal{F}_t\}\)-martingale under \(\tpr\).
\end{lemma}
\begin{pf}
By the self-financing property, we have
\begin{align*}
\odif{X_t}&=\sum_{i=1}^{m}\Delta_{t}^{(i)}\odif{S_{t}^{(i)}}+r_t
\underbrace{\left(X_t-\sum_{i=1}^{m}\Delta_{t}^{(i)}S_{t}^{(i)}\right)}
_{\text{non-stock component}}\odif{t}
=r_tX_t\odif{t}+\sum_{i=1}^{m}\Delta_{t}^{(i)}
\left(\odif{S_{t}^{(i)}}-r_tS_{t}^{(i)}\odif{t}\right) \\
&=r_tX_t\odif{t}+\sum_{i=1}^{m}\frac{\Delta_{t}^{(i)}}{D_t}
(D_t\odif{S_{t}^{(i)}}-S_{t}^{(i)}r_tD_t\odif{t})
=r_tX_t\odif{t}+\sum_{i=1}^{m}\frac{\Delta_{t}^{(i)}}{D_t}
(D_t\odif{S_{t}^{(i)}}+S_{t}^{(i)}\odif{D_t}) \\
\overset{\text{(It\^o product rule)}}&{=}
r_tX_t\odif{t}+\sum_{i=1}^{m}\frac{\Delta_{t}^{(i)}}{D_t}\odif{(D_tS_{t}^{(i)})}.
\end{align*}
Therefore, for the discounted portfolio value process \(\{D_tX_t\}\), by It\^o
product rule we have
\begin{align*}
\odif{(D_tX_t)}&=D_t\odif{X_t}+X_t\odif{D_t}+\underbrace{\odif{D_t,X_t}}_{0}
=D_t(\odif{X_t}-r_tX_t\odif{t}) \\
&=\sum_{i=1}^{m}\Delta_{t}^{(i)}\odif{(D_{t}S_{t}^{(i)})}.
\end{align*}
Under the risk-neutral measure \(\tpr\), by definition we know that
\(\{D_{t}S_{t}^{(i)}\}\) is a \(\{\mathcal{F}_t\}\)-martingale for every
\(i=1,\dotsc,m\). Hence, each \(\{D_{t}S_{t}^{(i)}\}\) has zero drift term (by
the converse of \labelcref{it:zero-drift-mart}), implying that \(\{D_tX_t\}\)
has zero drift term also. So, \(\{D_tX_t\}\) is a
\(\{\mathcal{F}_t\}\)-martingale under \(\tpr\).
\end{pf}
\item \textbf{Arbitrage.} The \emph{first fundamental theorem of asset pricing}
relates the concepts of risk-neutral measure and arbitrage. To prove the result
mathematically, first we need to give a mathematical definition of
\emph{arbitrage} (though we already have some understanding on what it means
from previous courses). An \defn{arbitrage} is a portfolio value process
\(\{X_t\}\) satisfying that
\begin{enumerate}[label={(\arabic*)}]
\item \emph{(No initial investment)} \(X_0=0\).
\item \emph{(Potential for earning profit without risk)} For some \(T>0\),
\(\prob{X_T\ge 0}=1\) and \(\prob{X_T>0}>0\).
\end{enumerate}
Basically, this is a mathematical way to express the idea that arbitrage is a
way to potentially earn profit without initial investment and without any risk,
essentially a \emph{free lunch}.

A criterion for the existence of arbitrage is the following.
\begin{proposition}
\label{prp:arbitrage-crit}
An arbitrage exists iff there is a portfolio price process \(\{Y_t\}\)
satisfying that
\begin{enumerate}[label={(\arabic*)}]
\item \emph{(Positive initial investment)} \(Y_0>0\).
\item \emph{(Earning at least risk-free rate and potentially earning more than
risk-free rate)} For some \(T>0\), \(\prob{Y_T\ge Y_0/D_T}=1\) and
\(\prob{Y_T>Y_0/D_T}>0\).
\end{enumerate}
\begin{note}
We have \(Y_0/D_T=Y_0e^{\int_{0}^{T}r_t\odif{t}}\). So, this condition means
that the portfolio earns at least the risk-free rate almost surely, and has a
positive probability to earn more than risk-free rate.
\end{note}
\end{proposition}
\begin{pf}
``\(\Rightarrow\)'': Assume that an arbitrage \(\{X_t\}\) exists. We can
construct a portfolio by investing \(Y_0>0\) in the risk-free bond and having a
long position in the portfolio corresponding to the arbitrage. Then the
resulting portfolio price process, denoted by \(\{Y_t\}\), satisfies two
conditions above.

``\(\Leftarrow\)'': Assume such portfolio price process \(\{Y_t\}\) exists. We
can construct a portfolio by having a long position in the portfolio
corresponding to such process, and borrowing \(Y_0>0\) at risk-free rate.
Then, the resulting portfolio price process, denoted by \(\{X_t\}\), satisfies
the two conditions for qualifying as an arbitrage.
\end{pf}
\item \textbf{First fundamental theorem of asset pricing.}
We are now ready to prove the first fundamental theorem of asset pricing,
which is about the \emph{existence} of risk-neutral measure.
\begin{theorem}[First fundamental theorem of asset pricing]
\label{thm:first-ftap}
If a risk-neutral measure \emph{exists} in a market (having the dynamics
suggested by \labelcref{it:market-dynamics}), then the market has no
arbitrage.
\end{theorem}
\begin{pf}
Assume that the market has a risk-neutral measure \(\tpr\). By
\Cref{lma:disc-port-price-mart-rn}, every discounted portfolio value process
\(\{D_tX_t\}\) is a \(\{\mathcal{F}_t\}\)-martingale under \(\tpr\), which
particularly implies that \(\texpv{D_TX_T}=D_0X_0=X_0\). Now, consider any
portfolio value process \(\{X_t\}\) with \(X_0=0\). We then have
\(\texpv{D_TX_T}=X_0=0\).

Now, assume to the contrary that \(\prob{X_T\ge 0}=1\) and \(\prob{X_T>0}>0\)
(so arbitrage exists). From this, we know that \(\prob{X_T<0}=0\). Since
\(\tpr\) and \(\pr\) are equivalent, we have \(\tprob{X_T<0}=0\),
and hence \(\tprob{X_T\ge 0}=1\). As we have \(\texpv{D_TX_T}=0\), this forces
that \(\tprob{X_T=0}=1\) and \(\tprob{X_T>0}=0\); otherwise, we would have
\(\texpv{D_TX_T}>0\). Applying the equivalence of \(\tpr\) and \(\pr\) again gives
\(\prob{X_T>0}=0\), contradiction.
\end{pf}

The first fundamental theorem of asset pricing provides a sufficient condition
for ensuring that our market model to be arbitrage-free, namely the existence
of risk-neutral measure. From the discussion in
\labelcref{it:multi-dim-rn-meas}, we know that the existence of solution to the
\emph{market price of risk equations} (existence of those
\(\{\Theta_{t}^{(j)}\}\)'s) implies that a risk-neutral measure exists, which in turn implies
that the market has no arbitrage, by the first fundamental theorem of asset
pricing. In short, the existence of solution to the market price of risk
equations is a guarantee for having no arbitrage in the market.
\item \textbf{Completeness of market.} As one may expect,
there is also a \emph{second} fundamental theorem of asset pricing. This time,
it relates the \emph{uniqueness} of risk-neutral measure and the possibility of
replicating derivatives. Prior to the discussion of the theorem, we first lay
some groundwork about the replication of derivatives.  A market is said to be
\defn{complete} if every derivative can be replicated (or hedged). Now, let us
consider what we need for the market to be complete, by utilizing an argument
similar to the one in \labelcref{it:one-stock-rep}.

Consider a market where the underlying filtration \(\{\mathcal{F}_t\}_{t\in
[0,T]}\) is the one generated by the \(d\)-dimensional Brownian motion
\(\{\vect{W}_t\}_{t\in [0,T]}\). Suppose that we have found a solution to the
market price of risk equations, thereby getting a risk-neutral measure \(\tpr\)
by applying the multidimensional Girsanov theorem (and implying the market is
arbitrage-free by the first fundamental theorem of asset pricing).

Consider any derivative with time-\(T\) payoff \(V_T\), which is
\(\mathcal{F}_T\)-measurable and integrable. Let
\(\widetilde{M}_t=\texpv{D_TV_T|\mathcal{F}_t}\) for all \(t\in[0,T]\). Since
\(\{\widetilde{M}_t\}\) is a Doob martingale under \(\tpr\), by
\Cref{thm:multi-dim-mart-rep} we have
\[
\widetilde{M}_T
=\widetilde{M}_0+\sum_{j=1}^{d}\int_{0}^{T}\widetilde{\Gamma}_t\odif{\widetilde{W}_t^{(j)}}
\]
where \(\{\widetilde{\Gamma}_t^{(1)}\},\dotsc,\{\widetilde{\Gamma}_t^{(d)}\} \)
are \(\{\mathcal{F}_t\}\)-adapted processes.

Next, consider any self-financing portfolio with time-\(t\) position on stock
\(i\) being \(\Delta_{t}^{(i)}\), where \(\{\Delta_{t}^{(i)}\}\) is
\(\{\mathcal{F}_t\}\)-adapted for each \(i=1,\dotsc,m\), and the non-stock
component being invested in the risk-free bond. Let \(X_t\) be the time-\(t\)
value of the portfolio.

From the proof of \Cref{lma:disc-port-price-mart-rn}, we have
\begin{align*}
\odif{(D_tX_t)}&=\sum_{i=1}^{m}\Delta_{t}^{(i)}\odif{(D_tS_{t}^{(i)})}
\overset{\text{\labelcref{it:multi-dim-rn-meas}}}{=}
\sum_{i=1}^{m}\Delta_{t}^{(i)}D_tS_{t}^{(i)}
\sum_{j=1}^{d}\sigma_{t}^{(ij)}\odif{\widetilde{W}_{t}^{(j)}}
=\sum_{j=1}^{d}\sum_{i=1}^{m}
\Delta_{t}^{(i)}D_{t}S_{t}^{(i)}\sigma_{t}^{(ij)}\odif{\widetilde{W}_{t}^{(j)}}.
\end{align*}
This implies that
\[
D_TX_T=X_0+\sum_{j=1}^{d}\int_{0}^{T}\sum_{i=1}^{m}
\Delta_{t}^{(i)}D_{t}S_{t}^{(i)}\sigma_{t}^{(ij)}\odif{\widetilde{W}_{t}^{(j)}}.
\]
Therefore, for this self-financing portfolio to replicate the derivative, we
need to set \(X_0=\widetilde{M}_0=\texpv{D_TV_T|\mathcal{F}_0}\), and the
positions \(\Delta_{t}^{(1)},\dotsc,\Delta_t^{(m)}\) such that
the \defn{hedging equations}
\[
\frac{\widetilde{\Gamma}_t^{(j)}}{D_t}
=\sum_{i=1}^{m}\Delta_{t}^{(i)}S_{t}^{(i)}\sigma_{t}^{(ij)},\quad j=1,\dotsc,d
\]
are satisfied for all \(t\in [0,T]\). Note that the hedging equations form a
system of \(d\) equations in \(m\) unknowns
\(\Delta_{t}^{(1)},\dotsc,\Delta_t^{(m)}\).  Unlike \Cref{prp:rep-port-exist},
here in general we cannot guarantee that there \emph{are} positions that solve
the hedging equations, and so generally we cannot ensure that every derivative
can be replicated, i.e., the market may not be complete. Nonetheless, the second
fundamental theorem of asset pricing gives us a criterion for the
completeness of the market, namely the \emph{uniqueness of risk-neutral measure}.
\item \textbf{Second fundamental theorem of asset pricing.}
\begin{theorem}[Second fundamental theorem of asset pricing]
\label{thm:second-ftap}
Consider a market (having the dynamics suggested by
\labelcref{it:market-dynamics}) that has a risk-neutral measure. The market is
complete iff the risk-neutral measure is unique.
\end{theorem}
\begin{pf}
(Sketch)
``\(\Rightarrow\)'': Assume the market is complete. Let \(\tpr_1\) and
\(\tpr_2\) be risk-neutral measures, and fix any
\(A\in\mathcal{F}_T\overset{\text{\labelcref{it:market-dynamics}}}{=}\mathcal{F}\).
Consider a derivative with payoff \(V_T=\indic_{A}/D_T\). Due to the
completeness of the market, this derivative can be replicated. Let \(\{X_t\}\)
be the value process of the replicating portfolio. By
\Cref{lma:disc-port-price-mart-rn}, the discounted portfolio value process
\(\{D_tX_t\}\) is a \(\{\mathcal{F}_t\}\)-martingale under both \(\tpr_1\) and
\(\tpr_2\). Hence, we have
\[
\tprobs{1}{A}=\texpvs{1}{D_TV_T}\overset{(X_T=V_T)}{=}
\texpvs{1}{D_TX_T}\overset{\text{(martingale)}}{=}D_0X_0
\overset{\text{(martingale)}}{=}
\texpvs{2}{D_TX_T}\overset{(X_T=V_T)}{=}
\texpvs{2}{D_TV_T}=\tprobs{2}{A}.
\]
It follows that \(\tpr_{1}=\tpr_{2}\), establishing the uniqueness.

``\(\Leftarrow\)'': Assume the risk-neutral measure \(\tpr\) is unique. It can
then be shown that the underlying filtration \(\{\mathcal{F}_t\}\) must be the
one generated by the \(d\)-dimensional Brownian motion \(\{\vect{W}_t\}\).
Knowing this, one can further show that the uniqueness of risk-neutral measure
forces that there is only one solution to the market
price of risk equations. Now, fix any \(t\in [0,T]\) and \(\omega\in\Omega\).
The market price of risk equations can be expressed as a system of linear
equations in matrix-vector form: \(A\vect{x}=\vect{b}\), where
\(A=[\sigma_{t}^{(ij)}(\omega)]_{i,j=1,1}^{m,d}\in\R^{m\times d}\),
\(\vect{x}=(\Theta_{t}^{(1)}(\omega),\dotsc,\Theta_{t}^{(d)}(\omega))\in\R^{d}\),
and \(\vect{b}=(\alpha_{t}^{(1)}(\omega)-r_t(\omega),\dotsc,
\alpha_{t}^{(m)}(\omega)-r_t(\omega))\in\R^{m}\). Then, we know that there is a
unique solution \(\vect{x}\) to the system.

In a similar way, we express the hedging equations as a system of linear
equations in matrix-vector form: \(A^{T}\vect{y}=\vect{c}\), where \(A\) is
defined above, \(\vect{y}=(y_1,\dotsc,y_m)
:=(\Delta_{t}^{(1)}(\omega)S_{t}^{(1)}(\omega),\dotsc,
\Delta_{t}^{(m)}(\omega)S_{t}^{(m)}(\omega))\in\R^{m}\), and
\(\vect{c}=(\widetilde{\Gamma}_{t}^{(1)}(\omega)/D_t
,\dotsc,\widetilde{\Gamma}_{t}^{(d)}(\omega)/D_t)\in\R^{d}\). Based on the
unique solution \(\vect{x}\), it can be shown that this system
has a solution \(\vect{y}\) for every \(\vect{c}\in\R^{d}\), meaning that there
always exists a solution to the hedging equations, and so every derivative can
be replicated, i.e., the market is complete.
\end{pf}

\item \textbf{Relationship between arbitrage/completeness and the market price
of risk equations.} Based on \labelcref{it:multi-dim-rn-meas} and the proof
sketch of \Cref{thm:second-ftap} above, we can deduce the following connection
between arbitrage/completeness and the existence/uniqueness of solution to the
market price of risk equations:
\begin{enumerate}
\item\label{it:sol-exist-iff-no-arbitrage} \emph{(Relating arbitrage and
existence of solution)} The market price of risk equations has a solution iff
the market has no arbitrage.
\item\label{it:sol-unique-imp-complete} \emph{(Relating completeness and
uniqueness of solution)} If the market price of risk equations have a unique
solution, then the market is complete.
\end{enumerate}
\end{enumerate}
